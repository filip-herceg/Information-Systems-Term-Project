\chapter{Feasibility Analysis}
\label{ch:feasibility-analysis}
The feasibility analysis aims to assess the viability of developing an online booking system for small businesses.
This analysis is divided into three main areas: technical feasibility, economic feasibility, and organizational feasibility.

\section{Technical Feasibility}
\label{sec:technical-feasibility}
\subsection*{Objective:}
\label{subsec:tf-objective}
Evaluate whether the current technology stack and team skills are sufficient to build the proposed online booking system.

\subsection*{Technology Stack:}
\label{subsec:tf-technology-stack}

\begin{itemize}
    \item \textbf{Frontend:} The frontent uses ASP.NET MVC, a powerful framework for building dynamic web applications, and Bootstrap for responsive design. This combination provides a robust foundation for creating a user-friendly interface that works across different devices\footcite[][]{missing}.
    \item \textbf{Backend:} The backend will be powered by ASP.NET Web API, providing robust APIs for service management, appointment booking, and user authentication.
    \item \textbf{Database:} SQL Server will be used to store user data, service information, and appointment details, ensuring data integrity and security.
    \item \textbf{Hosting Platform:} The system will primarily be developed on localhost, but can be extended to be hosted on Microsoft Azure, utilizing its cloud services for scalability, reliability, and security. Alternatively any other suitable cloud service can be used.
    \item 
\end{itemize}

\subsection*{Key Considerations:}
\label{subsec:tf-key-considerations}

\begin{itemize}
    \item \textbf{Team Expertise:} The team does not have prior experience with ASP.NET MVC and Web API, but has a some experience in C\# and .NET development. Training and resources are available to upskill the team in these technologies.
    \item \textbf{Development Tools:} Visual Studio provides robust tools for developing, debugging and deplying ASP.NET applications\footcite[][]{missing}, making it easier for the team to work efficiently.
    \item \textbf{Scalability:} ASP.NET and SQL Server are well-suited for scalable applications\footcite[][]{missing}, allowing the system to handle increased traffic and data as the business grows.
    \item \textbf{Security:} ASP.NET provides built-in security features, such as authentication and authorization, to protect user data and ensure secure access to the system and data protection\footcite[][]{missing}. This takes away the burden of implementing security from scratch and maintaining it.
\end{itemize}

\subsection*{Conclusion:}
\label{subsec:tf-conclusion}
The technical aspects of the projects are feasable, but due to the lack of experience in ASP.NET MVC and Web API, the team will need to upskill in these areas. The technology stack is well-suited for the project requirements and provides a solid foundation for building an efficient, secure, scalable, and user-friendly online booking system.

\section{Economic Feasibility}
\label{sec:economic-feasibility}

\subsection*{Objective:}
\label{subsec:ef-objective}
Assess the cost-effectiveness of the project, ensuring it aligns with the budget constraints.

\subsection*{Cost Factors:}
\label{subsec:ef-cost-factors}

\begin{itemize}
    \item \textbf{Development Costs:} Utilizing free and open-source tools minimizes development costs. ASP.NET and SQL Server offer free community editions that are sufficient for this project\footcite[][]{missing}\footcite[][]{missing}. Essentially the only cost will be the time spent by the team.
    \item \textbf{Hosting Costs:} Hosting the application on localhost, Azure or another cloud service using the free tier will eliminate hosting expenses during the development and initial deployment phases.
    The deployment for a small business will be minimal, and the costs can most likely be covered by the business.
    \item \textbf{Maintenance Costs:} The system will require regular maintenance and updates to ensure optimal performance and security. Usually this can be managed within the free service tiers of the development tools and hosting platforms, minimizing ongoing costs\footcite[][]{missing}. Small businesses will have to hire dedicated staff or outsource the maintenance, therefore increasing costs.
\end{itemize}

\subsection*{Benefit Analysis:}
\label{subsec:ef-benefit-analysis}
\begin{itemize}
    \item \textbf{Efficiency Gains:} Automating appointment scheduling reduces the time spent on manual bookings, allowing staff to focus on providing services, potentially increasing revenue.
    \item \textbf{Customer Satisfaction:} Providing an online booking system improves customer experience, making it easier for customers to book appointments, potentially increasing customer retention and loyalty.
    \item \textbf{Competitive Advantage:} Offering online booking sets the business apart from competitors, attracting new customers and retaining existing ones. Although due to the current market trends, this is becoming a necessity rather than a competitive advantage\footcite[][]{missing}. In future the lack of an online booking system could instead be a disadvantage.
\end{itemize}

\subsection*{Conclusion:}
\label{subsec:ef-conclusion}
The project is economically feasible as it leverages free tools and hosting options, ensuring development and maintenance costs remain low. The anticipated benefits in terms of efficiency, customer satisfaction, and competitive advantage justify the investment.

\section{Organizational Feasibility}
\label{sec:organizational-feasibility}

\subsection*{Objective:}
Evaluate the organizational capacity to support the development and deployment of the online booking system.

\subsection*{Stakeholder Involvement:}

\begin{itemize}
    \item \textbf{Business Owners:} Will provide requirements and feedback throughout the development process to ensure the system meets their needs.
    \item \textbf{Technical Team:} Composed of developers proficient in ASP.NET and SQL Server, responsible for the implementation and maintenance of the system.
    \item \textbf{End Users:} Clients who will use the system to book appointments. Their feedback will be crucial during user acceptance testing.
\end{itemize}

\subsection*{Organizational Support:}
\begin{itemize}
    \item \textbf{Management Support:} Full support from business management to transition from manual to digital appointment scheduling.
    \item \textbf{Training:} Minimal training required for staff to manage the system due to its user-friendly design. Training materials and sessions will be provided to ensure smooth adoption.
    \item \textbf{Change Management:} A clear plan will be in place to handle the transition, including communicating the benefits of the new system to all stakeholders and addressing any concerns promptly.
\end{itemize}

\subsection*{Conclusion:}
The organization is well-equipped to support the development, deployment, and adoption of the online booking system. With management support and a focus on user training, the transition is expected to be smooth, ensuring the system's successful implementation and utilization.

Summary

\section{Feasibility Analysis Summary}
\label{sec:feasibility-analysis-summary}
The feasibility analysis indicates that developing an online booking system for small businesses is technically, economically, and organizationally feasible. The chosen technology stack, cost-effective approach, and strong organizational support provide a solid foundation for the project's success.
